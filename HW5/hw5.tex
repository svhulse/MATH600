\documentclass[10pt,a4paper]{article}
\usepackage{amsmath}
\usepackage{amsfonts}
\usepackage{amssymb}

\newcommand{\N}{\mathbb{N}}
\newcommand{\R}{\mathbb{R}}

\title{MATH 600 Homework 5}
\author{Sam Hulse}

\begin{document}
\maketitle

\subsubsection{Problem 1}
Let $f: M \rightarrow N$ be a continuous function on $M$ and let $A$ be a sequentially compact set in $M$. Let $\{x_n\}$ be a sequence in $f(A)$. Since $f^{-1}f(A) \subseteq A$, $f^{-1}(x_n) \subseteq A$. Since $A$ is sequentially compact, there exits a convergent subsequence of $f^{-1}(x_n)$ in $A$ which coverges to a point $x_*$ in $A$. Since $f$ is continuous, $\lim_{i\to\infty} f^{-1}(x_{n_i}) = x_*$ implies that $f(f^{-1}(x_{n_i})) \subseteq f(x_*) \Rightarrow \{x_{n_i}\} = f(x_*)$. Since $x_* \in A$, $f(x_*) \in f(A)$, thus every sequence in $f(A)$ has a convergent subsequence which converges in $f(A)$.

\subsubsection{Problem 2}
Let $f: \R \rightarrow \R$ be continuous. Of the four functions\\
\begin{center}
(1): $\{x \in \R | f(x) = 0\}$\\
(2): $\{x \in \R | f(x) > 1\}$\\
(3): $\{x \in \R | f(x) \geq 0\}$\\
(4): $\{x \in \R | 0 \leq f(x) \leq 1\}$\\ 
\end{center}

\noindent Functions (1), (3), and (4) are closed\\
Functions (1) and (4) are compact\\
Function (2) is open\\
From what we are given, we can not necesarily say that any set is connected\\

\subsubsection{Problem 3}
Let $S^{n-1} := \{\mathbf{x} \in \R^n: ||x||_2 = 1\}$\\

(1):Since $S^{n-1}$ is the inverse image of $f(\mathbf{x}) = 1$, the image of $f$ is the singleton $\{1\}$. Since the singleton set is closed, the reverse image of $\{1\}$ under $f$ is closed as well, therefore $S^{n-1}$ is closed. (Prove that f is continuous)\\

(2): We know from (1) that $S^{n-1}$ is closed, and since we are in a subset of the Euclidean space, we must now show it is also bounded to show compactness. Let $\mathbf{x}, \mathbf{y} \in S^{n-1}$. Then $||x|| = ||y|| = 1 \Rightarrow ||x|| + ||y|| = 2 \Rightarrow ||x|| + ||-y|| = 2$ (since $||-y|| = |-1|||y|| = ||y||$). Therefore, $||x - y|| \leq ||x|| + ||-y|| = 2$. This shows that $d(x, y) \leq 2$ for any $x, y \in S^{n-1}$, thus $S^{n-1}$ is bounded, so that $S^{n-1}$ is compact.\\

(3): If $x \in S^{n-1} \Rightarrow ||x||_2 = 1 \Rightarrow \sqrt{x_1^2 + x_2^2 +...+ x_n^2} = 1 \Rightarrow x_1^2 + x_2^2 +...+ x_n^2 = 1 \Rightarrow x_i \leq 1 \ \forall i \leq n \Rightarrow x_i^p \leq 1 \ \forall i \leq n, \ \forall p \in \N$. Therefore $\sum_{i = 1}^n x_i^p \leq n \ \forall\ \N$. Since $|x|^{1/p} \leq x \ \forall x \geq 0, (\sum_{i = 1}^n |x_i|^p)^{1/p} \leq n$. Therefore $||x||_p$ is bounded above, and since $||x||_p \in \R$, the supremum exists.\\

(4):\\

(5):\\

\subsubsection{Problem 4}
Let $f: \R^n \rightarrow \R^n$ be continuous on $\R^n$. Define the function $g: \R^n \rightarrow \R^n \times R^m$ as:
\centerline{$g(x):= (x, f(x)), \ \forall x \in \R^n$}\\

(1): Let $\{\mathbf{x_n}\}$ be a convergent sequence in $\R^n$ which converges to a point $\mathbf{x_*}$. Then $g(\mathbf{x_n}) = (\mathbf{x_n}, f(\mathbf{x_n}))$. The first function, the identity function is trivialy convergent since $\{\mathbf{x_n}\} \rightarrow \mathbf{x_*}$. We know that the second function, $f(\mathbf{x})$ is continuous, so then $f(\mathbf{x_n})$ converges to $f(\mathbf{x_*})$\\

\noindent (2): Let $S := \{(x, f(x))|x \in \R^n\} \subseteq \R^n \times \R^m$. Since we showed in (1) that $g(x)$ is continuous for all $x \in \R^n$, and $\R^n$ is connected, then the image of $\R^n$ under $g(x)$ is also connected, therefore $S$ is connected. To show that the $S$ is closed, I will use the sequentiall critereon for closed sets. Let $\{x_n, f(x_n)\}$ be a convergent sequence in $S$ which converges to the point $\{x, y\}$. Since $f(x_n) \rightarrow f(x), f(x) = y$. Therefore $\{x_n, f(x_n)\} \rightarrow (x, f(x))$. Because $(x, f(x)) \in S$, the limit of any sequence in $S$ is contained in S. Therefore, $S$ is closed.


\subsubsection{Problem 5}
Let $A$ and $B$ be two path-connected sets in a metic space such that $A \cap B$ is nonempty. Show that $A \cup B$ is path-connected.\\

Two show that $A \cup B$ is path-connected, we must show that for any $x, y \in A \cup B$, there exits a continuous function $h(x): [0, 1] \rightarrow A \cup B$ such that $h(0) = x$, and $h(1) = y$. There are then four cases to consider. The first case if that $x, y \in A \subseteq A \cup B$. Since $A$ is path connected, this case is trivial. The second case if both $x, y \in B$, and once again, the path connectedness of B demonstrates that we can find a path connecting $x$ and $y$.

Now consider the case that $x \in A$ and $y \in B$. Let $z \in A \cap B$. We know this point exists because $A$ and $B$ have a non-empty intersection. Then for every $x \in A$ and every $y \in B$, there exists a continuous function $f: [0, 1] \rightarrow A$ such that $f(0) = x$ and $f(1) = z$ and a continuous function $g: [0, 1] \rightarrow B$ such that $g(0) = y$ and $g(1) = z$. Then define $h : [0, 1] \rightarrow A \cup B$ as:\\


\centerline{$h(x) = \begin{cases} 
      f(2x) & 0 \leq x/2 \leq 1/2\\
      g(2x - 1) & 1/2 < x/2 \leq 1\\ 
   \end{cases}$}
\vspace{1pc}
Then we have $h(0) = x$ and $h(1) = y$. Additionally, since $h(x)$ is pointwise continous on $[0, 1]$, $g(x)$ is pointwise continuous on $[0, 1]$ and $f(1) = g(0)$, $h(x)$ is continous on $[0, 1]$.

The case where $x \in B$ and $y \in A$ can be constructed analogously to the previous case. This shows that for any $x, y \in A \cup B$, there exists a continuous function $f: [0, 1] \rightarrow A \cup B$ such that $f(0) = x$ and $f(1) = y$. Therefore $A \cup B$ is path-connected. 

\end{document}

